% Created 2021-09-01 Wed 00:45
% Intended LaTeX compiler: pdflatex
\documentclass[a4paper,12pt]{article}
\usepackage[utf8]{inputenc}
\usepackage[T1]{fontenc}
\usepackage{graphicx}
\usepackage{grffile}
\usepackage{longtable}
\usepackage{wrapfig}
\usepackage{rotating}
\usepackage[normalem]{ulem}
\usepackage{amsmath}
\usepackage{textcomp}
\usepackage{amssymb}
\usepackage{capt-of}
\usepackage{hyperref}
\author{\href{mailto:manuel.fuica.morales@gmail.com}{manuel.fuica.morales@gmail.com}}
\date{September 2021}
\title{Rodrigo Hidalgo Barra's Ejercicios Resueltos de Calculo Integral (English)\\\medskip
\large Orgmode Adaptation}
\hypersetup{
 pdfauthor={\href{mailto:manuel.fuica.morales@gmail.com}{manuel.fuica.morales@gmail.com}},
 pdftitle={Rodrigo Hidalgo Barra's Ejercicios Resueltos de Calculo Integral (English)},
 pdfkeywords={},
 pdfsubject={},
 pdfcreator={Emacs 27.1 (Org mode 9.5)}, 
 pdflang={Spanish}}
\begin{document}

\maketitle
\setcounter{tocdepth}{3}
\tableofcontents


\section{Abstract}
\label{sec:orgf2d7d35}

This is a journal of calculus problems made to practice calculus notation on
Latex, based on Rodrigo Hidalgo Barra's Ejercicios Resueltos de Cálculo
Integral (Solved Integral Calculus Problems); handbook that has been added
a chapter of derivatives post publishing.

\section{Introduction}
\label{sec:orgc270800}

This document's object is to support the material seen in classes by
handling a variety of integral calculus and infinite series problems.
Despite theorems and formulae are briefed they are not demonstrated, they
are only applied without further analysis.

The first chapter possesses 120 undefined integrals solved by classic
different integration methods. Second chapter accounts The Two Fundamental
Calculus Theorems and their applications. Third chapter consists of
problematic integrals concerning improper integrals, convergence criteria
and beta/gamma functions. Lastly, in the fourth chapter Infinite Series
and their convergence criteria are approached.

[Note that a chapter of derivatives has been added to this particular
handbook after purchase, chapter that has been placed first thing in order
to correspond the way it is taught in the first semesters of a regular
engineering degree].

\hfill Adaption of actual handbook's Introduction.

\hfill May 26, 2019, Sunday.

\newpage

\section{Derivative}
\label{sec:orgf955146}
\subsection{Definition}
\label{sec:org5e4c934}
\subsection{Semi Derivative}
\label{sec:org47837b8}
\subsection{Derivability and Continuity}
\label{sec:org58a7738}
\subsection{High Order Derivatives}
\label{sec:org1c71cf7}
\subsection{Interpretation of a Derivative}
\label{sec:org6a07de3}
\subsection{Basic Derivation Rules}
\label{sec:org9766280}
\subsection{Implicit Derivatives}
\label{sec:org3a91aca}
\subsection{Exponential and Logarithmic Derivatives}
\label{sec:orgcab42b7}
\subsection{Trigonometric Derivatives}
\label{sec:org9434db5}
\subsection{Inverse Trigonometric Derivatives}
\label{sec:org4842586}
\subsection{Parametric Derivatives}
\label{sec:org5edea0c}
\subsection{L'H$\backslash$\textsuperscript{opital}'s Rule}
\label{sec:org372b69e}
\subsection{Solved Problems}
\label{sec:org1e39301}
\subsection{Applications}
\label{sec:orgfaec3bc}
\subsection{Critic Point}
\label{sec:org5d14c9b}
\subsection{Growth and Decay}
\label{sec:orgfbd09a9}
\subsection{Concavity}
\label{sec:org1957cd2}
\subsection{Local and Relative Extrema}
\label{sec:org4e872c7}
\subsection{Absolute Maximum and Minimum}
\label{sec:orgde63d6c}
\subsection{Asymptote}
\label{sec:org6c95a58}
\subsection{Solved Problems}
\label{sec:org794ad8e}

\newpage

\section{Indefinite Integral}
\label{sec:orgc07b5ce}
\subsection{Differential}
\label{sec:org5f608cc}
\subsection{Direct}
\label{sec:orga7e8ebe}
\subsection{Substitution}
\label{sec:org5277ea5}
\subsection{By Parts}
\label{sec:orgb6e5db8}
\subsection{Trigonometric Functions}
\label{sec:org309dc63}
\subsection{Trigonometric Substitution}
\label{sec:org14587bb}
\subsection{Partial Fractions}
\label{sec:org7efd929}
\subsection{Irrational Functions}
\label{sec:org75f9e88}
\subsection{t=tg(\$\(\theta\)\$$\backslash$,/2)}
\label{sec:orgcf214a6}
\subsection{Solved Problems}
\label{sec:org5ab093e}

\newpage

\section{Applications}
\label{sec:orge3e9222}
\subsection{Riemann Sum}
\label{sec:org96b9807}
\subsection{First Fundamental Theorem of Calculus}
\label{sec:orgea4d906}
\subsection{Second Fundamental Theorem of Calculus}
\label{sec:orgc2eac28}
\subsection{Mean Value Theorem}
\label{sec:orgb4e1a70}
\subsection{Area}
\label{sec:orgcd264fa}
\subsection{Arc Length}
\label{sec:orgdf88967}
\subsection{Solid of Revolution}
\label{sec:org8a1435e}
\subsection{Surface of Revolution}
\label{sec:orgca1affc}
\subsection{Momentum - Centroid}
\label{sec:org273d8ac}
\subsection{Pappus's Theorem}
\label{sec:orgbf63804}

\newpage

\section{Improper Integral}
\label{sec:orgbe8be4e}
\subsection{Convergence Definition}
\label{sec:orgfa6b806}
\subsection{Convergence Criteria}
\label{sec:org0562ea5}
\subsection{Gamma Function}
\label{sec:orge3cb25c}
\subsection{Beta Function}
\label{sec:org27a750f}
\section{Infinite Series}
\label{sec:orgf3b64dd}
\subsection{Definition}
\label{sec:orgb01972d}
\subsection{Infinite Series}
\label{sec:org72ba9fe}
\subsection{Geometric Series}
\label{sec:org68a3bc3}
\subsection{Telescoping Series}
\label{sec:orgf498b86}
\subsection{Convergence Criteria}
\label{sec:org356c24a}
\subsection{Convergence of a Series}
\label{sec:org30eda1d}
\end{document}
